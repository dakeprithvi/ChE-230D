\documentclass{article}
\usepackage[margin=2.54cm]{geometry}
\usepackage{color}
\usepackage{graphicx}
\usepackage{float}
\usepackage{polynom}
\usepackage{placeins}
\usepackage{longtable}

\DeclareOldFontCommand{\rm}{\normalfont\rmfamily}{\mathrm}
\DeclareOldFontCommand{\it}{\normalfont\rmfamily}{\mathrm}
\usepackage{mathtools}


\usepackage{amsmath, bm}   % Extra math commands and environments from the AMS
\usepackage{amssymb}   % Special symbols from the AMS
\usepackage{amsthm}    % Enhanced theorem and proof environments from the AMS
\usepackage{latexsym}  % A few extra LaTeX symbols
\usepackage{rxn}
\RequirePackage{booktabs}
\RequirePackage{units_jbr}
\RequirePackage{mpcsymbols}

\usepackage{url}
\providecommand{\email}{}
\renewcommand{\email}[1]{\texttt{#1}}

\usepackage[authoryear,round]{natbib}

\newtheorem{theorem}{Theorem}
\newtheorem{lemma}[theorem]{Lemma}
\newtheorem{proposition}[theorem]{Proposition}
\newtheorem{conjecture}[theorem]{Conjecture}
\newtheorem{corollary}[theorem]{Corollary}

\theoremstyle{definition}
\newtheorem{definition}[theorem]{Definition}
\newtheorem{remark}[theorem]{Remark}
\newtheorem{assumption}[theorem]{Assumption}
\newtheorem{hypothesis}[theorem]{Hypothesis}
\newtheorem{property}[theorem]{Property}

%\newcommand{\tL}{\tilde{L}}

\graphicspath{{./}{./figures/}{./figures/paper/}}

\setcounter{topnumber}{2}              %% 2
\setcounter{bottomnumber}{1}           %% 1
\setcounter{totalnumber}{3}            %% 3
\renewcommand{\topfraction}{0.9}       %% 0.7
\renewcommand{\bottomfraction}{0.9}    %% 0.3
\renewcommand{\textfraction}{0.1}      %% 0.2
\renewcommand{\floatpagefraction}{.7}  %% 0.5

\title{Structured non-linear hybrid model - ChE 230D}
\author{Prithvi Dake\\
Department of Chemical Engineering\\
University of California, Santa  Barbara\\
Santa Barbara, CA 93106}
\date{\today}

\begin{document}

\maketitle
The project is specifically aimed at model indentification applied to chemical plants. Here, we show a simplified `hybrid' modelling approach using neural networks to represent the difficult-to-model parts in the first-principles implementation \citep{kumar:rawlings:2023a}. We end the presentation with quantile regression to make the model selectively learn a certain quantile of the data which can then be used for uncertainty prediction.\\
\newline
\textbf{TOC:}
\section{Incentive for deep learning (specifically hybrid modelling)}
\section{Case study for partial state measurement}
\section{Towards a structured `greybox' model}
\section{Quantile regression for uncertainty prediction}



\bibliographystyle{abbrvnat}
\bibliography{abbreviations,articles,books,unpub,proceedings}
\end{document}
